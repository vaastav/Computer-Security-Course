\documentclass[11pt]{article}

\begin{document}

\centerline{\textbf{CPSC 490: Introduction to Computer Security - Course Proposal - SAMPLE DRAFT}}

\section{Course Rationale}

We would like to propose a Computer Security Course. Computer Security is an important part of Computer Science, and we feel that the theoretical aspects of Computer Security needs to be taught and discussed in a dedicated course at UBC. We feel that because of an increasing focus on safe programming principles and data security, computer security has become a necessary aspect of a software engineering job. Our course aims to provide an in-depth introduction to both theoretical and practical aspects of computer security and all the relevant topics associated with it. We will focus on the different kind of attacks on the different resources and the defenses against such attacks. 

This type of a computer security course has been offered in other universities with great success, such as CS 155 at Stanford and 6.858 at MIT to name a few.

\section{Prerequisites}

The prerequisites for this course are CPSC 313 and CPSC 317. No prior knowledge of computer security is required; however an introductory course in Probability(STAT 241/STAT 302) is definitely a pre-requisite for this course. We intend to introduce all the relevant topics from scratch.

\section{Format}

As with other offerings of CPSC 490, the course will be offered in the format of a seminar. Each week, we(the coordinators) will present a topic for discussion(eg. Autehntication, Web Security, etc.) and outline the relevant topics associate with it. The discussion may be led by the coordinators or the students themselves. Following the discussion, the coordinators will present a few problems related to the topic. The problems presented will be non-trivial, in that additional reading and research will be required before one of the discussed topics can be used to solve those problems. Each student will also have a chance to present problems of their choosing along with their solutions, as long as it pertains to the weekly topic. At the end of every week, we will assign homework problems in the format described in Section 4. The maximum enrolment of this course will be 15 students. 

\section{Homework Problems}

Each homework will consist of two different parts. The first part would be the theory part which would consist of some theoretical questions. The second part would be the implementation part which would be structured as a CTF contest and would require the students to implement the things learnt in class in order to get the answer to the problem. Each problem in the CTF contest would have a flag which would serve as the answer and this flag would only be found once the student has "broken". Such CTF contests are very popular and are similar to the ACM programming contests. The judge for such an online contest would be trivial as it would just compare the students' answer to the actual answer. A report for each contest would also be mandatory to hand in which would detail the process with which the student solved the problem and found the flag.

\section{Syllabus}

The seminar will attempt to cover the following topics with * denoting a topic that may or may not be covered (depending on the speed of the class). Topics may be added or removed depending on the class background and interests.

\begin{enumerate}
	\item \textbf{Ciphers}: Stream Ciphers,Block Ciphers,Modes of Operation.
	\item \textbf{Data Security} : \begin{enumerate}
			\item \textbf{Data Confidentiality} : Symmetric Encryption, Assymetric Encryption, Public Key Encryption (RSA, Probabilistic) 
			\item \textbf{Data Integrity} : Cryptographic Hash Functions
	   \end{enumerate}
   \item \textbf{Operating Systems Security} : Principle of Least Privilege, Isolation, Access Control
   \item \textbf{Network Security} : VPN, Firewalls, Intrusion Detection System, DNS Spoofing, DNS Rebinding Attack, ARP Spoofing, DOS, DDOS 
   \item \textbf{Web Security} : SQL Injections, Cross Site Scripting, Cross-Site Request Forgery, Same Origin Policy, HTTPS*
   \item \textbf{Code Integrity} : Stack Smash, Code Reuse, Heap Spraying, Binary Exploitation* 
   \item \textbf{Cloud Security} : Virtualization, Multi-tenancy, Centralized Target, Timing/Side channel attacks
   \item \textbf{Mobile Security} : Mobile Security Models(Android and iOS), Closed vs Open*, Permissions Fatigue*
   \item \textbf{Post-Quantum Crypto*}
\end{enumerate}

\section{Potential Faculty Sponsor}

\begin{enumerate}
   \item Ivan Beschastnikh
   \item Andrew Warfield
\end{enumerate}

\section{Quaification of the Coordinators}

We have 2 confirmed coordinators.

Vaastav Anand: 3rd (4th in UBC) year Honors in Computer Science and Software Engineering

Radu Nesiu : 4th year Computer Science Student.

\end{document}
